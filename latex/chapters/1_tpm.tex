\section{Tissue Probability Maps}

% Using the 10 tissue segmentations and their associated MR images, create a groupwise mean template and associated tissue probability maps that are representative of this population. Describe the process and justify your choice of transformation model (rigid, affine and/or non-rigid) [15].


Tissue Probability Maps (TPMs) of cerebrospinal fluid (CSF), grey matter (GM), white matter (WM) and background must be generated so that they can be used as spatial priors for tissue segmentation. The provided group of subjects with an associated label map representing a tissue segmentation has been used to generate the TPMs.


The TPMs are calculated as an average of all the label maps that have been projected to a common space defined by a \textit{template} image. An example of such a template is the one built at the Montreal Neurological Institute (MNI) \cite{evans_3d_1993}. However, for this project a template space has been built iteratively using a groupwise coregistration of all the subjects in the group. All the registrations have been performed using NiftyReg \cite{modat_global_2014}\footnote{The repository containing the Python code used for the experiments (over 2000 lines) and the \LaTeX\ code used to create this report is available online on GitHub, under request to the author.}.


\subsection{Interpolation}
The following interpolation methods have been used when applying a transform to an image:

\subsubsection{TPMs}
Linear interpolation has been used to resample a transformed TPM. This ensures that the values that were originally between 0 and 1 in the source space remain in that range once interpolated into the target space.

\subsubsection{$T_1$-weighted images} To resample anatomical images, sinc interpolation has been used to minimise the presence of aliasing. Modifying the original values range is not too relevant for these images.

\subsubsection{Label maps} Since the discrete values of the labels must be preserved, nearest-neighbour interpolation has been used to resample the label maps.



\subsection{Similarity measure}
The linear version of NiftyReg, the block-matching algorithm Aladin, uses normalised cross-correlation (NCC) as similarity measure. This metric is appropriate for the registrations performed in this project, since they are all monomodal, i.e. a pairwise linear relationship between image intensities is expected. If registrations were multimodal, e.g. MRI to CT, a different similarity measure as normalised mutual information or joint entropy would have been necessary.



\subsection{Rigid coregistration}
First, the $T_1$-weighted image of one segmented subject is taken as reference and the rest of the subjects are registered to it one by one using rigid registration, i.e. 6 degrees of freedom (DOF) representing 3D translation and rotation. The floating images are resampled to the reference space and averaged in order to generate the initial version of the template image. Since all the images values were in the same range ([0, 255]), no preprocessing was applied to the data. If that was not the case, the intensity of all the images should have been normalised before the averaging. For visualisation purposes, the label maps have also been transformed to the reference space and the results have been averaged label by label, yielding an initial version of the TPMs. Figure \ref{fig:template-rigid} shows the results of this initial coregistration step.

\begin{figure}
  \centering
  \includegraphics[width=\textwidth]{figures/rigid_template_collage}
  \caption{Results of the rigid coregistration. Top left: mean of the $T_1$-weighted images; top right: TPM of the CSF; bottom left: TPM of the GM; bottom right: TPM of the WM.}
  \label{fig:template-rigid}
\end{figure}

Using only rigid registration for the initial reduces the bias of the template. If the first coregistration were performed using affine registrations (12 DOF representing translation, rotation, shearing and anisotropic scaling), the consecutive iterations would be biased towards the reference image.


\subsection{Iterative affine coregistration}
Once a rigid version of the template is created, it can be used as reference for the following iterations of the coregistration. At each iteration, the transform from the previous step was used for initialisation. This iterative process generated a template image and TPMs that were sharper for each iteration, but did not converge. Up to 10 iterations, the smoothness reduction stopped being visually perceptible. However, the image seemed to diverge towards a sheared version of the template after 10 iterations. Figure \ref{fig:template-sheared} shows a comparison between the template image after 10 and 20 iterations and Figure \ref{fig:template-final} shows the final template after 10 affine iterations.


\begin{figure}
  \includegraphics[width=0.5\textwidth]{figures/template_sheared}
  \centering
  \caption{Final template after 10 affine iterations. Top left: mean of the $T_1$-weighted images; top right: TPM of the CSF; bottom left: TPM of the GM; bottom right: TPM of the WM.}
  \label{fig:template-sheared}
\end{figure}


\begin{figure}
  \centering
  \includegraphics[width=\textwidth]{figures/affine_template_iter_9_collage}
  \centering
  \caption{Template image after 10 affine registration iterations (green) blended with template image after 20 iterations (magenta). The shearing effect is especially visible at the top left and bottom right parts of the image.}
  \label{fig:template-final}
\end{figure}


Moving forward into running coregistration iterations using a free-form registration would result into sharper TPMs, but this might lead to a poorer tissue segmentation, since the priors might not be general enough. For example, if a very sharp template is registered to the subject using a strong regularisation, it is possible that a voxel of GM in the subject space corresponds to a priors voxel with very low GM probability, since it is hard to reach a perfect match between GM structures using intensity registration only. In that case, the voxel would probably be misclassified as WM (if it is bright) or CSF (if it is dark). Using affine registration only makes the TPMs smoother, leading to a better adaptability of the model to individual cases. Figure \ref{fig:template-std} shows the standard deviation of the final template across the 10 subjects registered to it. The higher variability in the grey matter means that TPMs should be smooth enough to generalise properly. A Gaussian blur image filter may also be used to obtain smoother TPMs, that might generalise even better.

\begin{figure}
  \includegraphics[width=\textwidth]{figures/affine_9_std}
  \centering
  \caption{Standard deviation across the 10 segmented subjects registered to the final template.}
  \label{fig:template-std}
\end{figure}
